% Options for packages loaded elsewhere
\PassOptionsToPackage{unicode}{hyperref}
\PassOptionsToPackage{hyphens}{url}
%
\documentclass[
]{article}
\usepackage{amsmath,amssymb}
\usepackage{lmodern}
\usepackage{iftex}
\ifPDFTeX
  \usepackage[T1]{fontenc}
  \usepackage[utf8]{inputenc}
  \usepackage{textcomp} % provide euro and other symbols
\else % if luatex or xetex
  \usepackage{unicode-math}
  \defaultfontfeatures{Scale=MatchLowercase}
  \defaultfontfeatures[\rmfamily]{Ligatures=TeX,Scale=1}
\fi
% Use upquote if available, for straight quotes in verbatim environments
\IfFileExists{upquote.sty}{\usepackage{upquote}}{}
\IfFileExists{microtype.sty}{% use microtype if available
  \usepackage[]{microtype}
  \UseMicrotypeSet[protrusion]{basicmath} % disable protrusion for tt fonts
}{}
\makeatletter
\@ifundefined{KOMAClassName}{% if non-KOMA class
  \IfFileExists{parskip.sty}{%
    \usepackage{parskip}
  }{% else
    \setlength{\parindent}{0pt}
    \setlength{\parskip}{6pt plus 2pt minus 1pt}}
}{% if KOMA class
  \KOMAoptions{parskip=half}}
\makeatother
\usepackage{xcolor}
\usepackage[margin=1in]{geometry}
\usepackage{color}
\usepackage{fancyvrb}
\newcommand{\VerbBar}{|}
\newcommand{\VERB}{\Verb[commandchars=\\\{\}]}
\DefineVerbatimEnvironment{Highlighting}{Verbatim}{commandchars=\\\{\}}
% Add ',fontsize=\small' for more characters per line
\usepackage{framed}
\definecolor{shadecolor}{RGB}{248,248,248}
\newenvironment{Shaded}{\begin{snugshade}}{\end{snugshade}}
\newcommand{\AlertTok}[1]{\textcolor[rgb]{0.94,0.16,0.16}{#1}}
\newcommand{\AnnotationTok}[1]{\textcolor[rgb]{0.56,0.35,0.01}{\textbf{\textit{#1}}}}
\newcommand{\AttributeTok}[1]{\textcolor[rgb]{0.77,0.63,0.00}{#1}}
\newcommand{\BaseNTok}[1]{\textcolor[rgb]{0.00,0.00,0.81}{#1}}
\newcommand{\BuiltInTok}[1]{#1}
\newcommand{\CharTok}[1]{\textcolor[rgb]{0.31,0.60,0.02}{#1}}
\newcommand{\CommentTok}[1]{\textcolor[rgb]{0.56,0.35,0.01}{\textit{#1}}}
\newcommand{\CommentVarTok}[1]{\textcolor[rgb]{0.56,0.35,0.01}{\textbf{\textit{#1}}}}
\newcommand{\ConstantTok}[1]{\textcolor[rgb]{0.00,0.00,0.00}{#1}}
\newcommand{\ControlFlowTok}[1]{\textcolor[rgb]{0.13,0.29,0.53}{\textbf{#1}}}
\newcommand{\DataTypeTok}[1]{\textcolor[rgb]{0.13,0.29,0.53}{#1}}
\newcommand{\DecValTok}[1]{\textcolor[rgb]{0.00,0.00,0.81}{#1}}
\newcommand{\DocumentationTok}[1]{\textcolor[rgb]{0.56,0.35,0.01}{\textbf{\textit{#1}}}}
\newcommand{\ErrorTok}[1]{\textcolor[rgb]{0.64,0.00,0.00}{\textbf{#1}}}
\newcommand{\ExtensionTok}[1]{#1}
\newcommand{\FloatTok}[1]{\textcolor[rgb]{0.00,0.00,0.81}{#1}}
\newcommand{\FunctionTok}[1]{\textcolor[rgb]{0.00,0.00,0.00}{#1}}
\newcommand{\ImportTok}[1]{#1}
\newcommand{\InformationTok}[1]{\textcolor[rgb]{0.56,0.35,0.01}{\textbf{\textit{#1}}}}
\newcommand{\KeywordTok}[1]{\textcolor[rgb]{0.13,0.29,0.53}{\textbf{#1}}}
\newcommand{\NormalTok}[1]{#1}
\newcommand{\OperatorTok}[1]{\textcolor[rgb]{0.81,0.36,0.00}{\textbf{#1}}}
\newcommand{\OtherTok}[1]{\textcolor[rgb]{0.56,0.35,0.01}{#1}}
\newcommand{\PreprocessorTok}[1]{\textcolor[rgb]{0.56,0.35,0.01}{\textit{#1}}}
\newcommand{\RegionMarkerTok}[1]{#1}
\newcommand{\SpecialCharTok}[1]{\textcolor[rgb]{0.00,0.00,0.00}{#1}}
\newcommand{\SpecialStringTok}[1]{\textcolor[rgb]{0.31,0.60,0.02}{#1}}
\newcommand{\StringTok}[1]{\textcolor[rgb]{0.31,0.60,0.02}{#1}}
\newcommand{\VariableTok}[1]{\textcolor[rgb]{0.00,0.00,0.00}{#1}}
\newcommand{\VerbatimStringTok}[1]{\textcolor[rgb]{0.31,0.60,0.02}{#1}}
\newcommand{\WarningTok}[1]{\textcolor[rgb]{0.56,0.35,0.01}{\textbf{\textit{#1}}}}
\usepackage{graphicx}
\makeatletter
\def\maxwidth{\ifdim\Gin@nat@width>\linewidth\linewidth\else\Gin@nat@width\fi}
\def\maxheight{\ifdim\Gin@nat@height>\textheight\textheight\else\Gin@nat@height\fi}
\makeatother
% Scale images if necessary, so that they will not overflow the page
% margins by default, and it is still possible to overwrite the defaults
% using explicit options in \includegraphics[width, height, ...]{}
\setkeys{Gin}{width=\maxwidth,height=\maxheight,keepaspectratio}
% Set default figure placement to htbp
\makeatletter
\def\fps@figure{htbp}
\makeatother
\setlength{\emergencystretch}{3em} % prevent overfull lines
\providecommand{\tightlist}{%
  \setlength{\itemsep}{0pt}\setlength{\parskip}{0pt}}
\setcounter{secnumdepth}{-\maxdimen} % remove section numbering
\ifLuaTeX
  \usepackage{selnolig}  % disable illegal ligatures
\fi
\IfFileExists{bookmark.sty}{\usepackage{bookmark}}{\usepackage{hyperref}}
\IfFileExists{xurl.sty}{\usepackage{xurl}}{} % add URL line breaks if available
\urlstyle{same} % disable monospaced font for URLs
\hypersetup{
  pdftitle={07\_Compare\_SIC\_to\_Sea\_Ice\_Charts},
  pdfauthor={Ellie Honan},
  hidelinks,
  pdfcreator={LaTeX via pandoc}}

\title{07\_Compare\_SIC\_to\_Sea\_Ice\_Charts}
\author{Ellie Honan}
\date{2023-03-07}

\begin{document}
\maketitle

\hypertarget{comparison-of-passive-microwave-sea-ice-concentration-data-to-sea-ice-chart-data.}{%
\section{Comparison of passive microwave sea ice concentration data to
sea ice chart
data.}\label{comparison-of-passive-microwave-sea-ice-concentration-data-to-sea-ice-chart-data.}}

Here, I am comparing sea ice concentration readings from 2 sources:
Passive Microwave data from AMSR2, and Sea Ice Charts from AARI
available from one day in my tracked period.

I have subset my trips to retain the points from January 21st 2022 only,
and have both types of sea ice data from the same date. I extracted the
AMSR2 data using the \texttt{raster} function \texttt{extract} and the
sea ice chart data was extracted in QGIS.

\hypertarget{sea-ice-chart-for-2022-01-22}{%
\subsection{Sea Ice Chart for
2022-01-22:}\label{sea-ice-chart-for-2022-01-22}}

\href{http://ice.aari.aq/antice/2022/01/20220121_nic/nic_antice_20220121_ct.png}{\includegraphics{Trip_Covariates/Sea_Ice_Chart_Plots/nic_antice_20220121_ct.png}}

This chart provides information on ice type, and is based on AMSR2 data
combined with satellite imagery.

\hypertarget{sea-ice-concentration-from-amsr2-for-the-same-date}{%
\subsection{Sea ice concentration from AMSR2 for the same
date:}\label{sea-ice-concentration-from-amsr2-for-the-same-date}}

\includegraphics{Trip_Covariates/Sea_Ice_Chart_Plots/AMSR2_2022_01_22.png}

\hypertarget{set-up}{%
\subsubsection{Set up:}\label{set-up}}

\begin{Shaded}
\begin{Highlighting}[]
\CommentTok{\#read in trips}
\NormalTok{comp }\OtherTok{\textless{}{-}} \FunctionTok{read.csv}\NormalTok{(}\StringTok{"C:/Users/wclp88/OneDrive {-} Durham University/R\_Projects/Honan\_SNPE\_Tracking\_2023/Trip\_Covariates/07\_Q\_Amsr2\_vs\_charts.csv"}\NormalTok{)}

 \CommentTok{\#what birds}
\NormalTok{birdSV }\OtherTok{=} \FunctionTok{unique}\NormalTok{(comp}\SpecialCharTok{$}\NormalTok{BIRDID)}
\end{Highlighting}
\end{Shaded}

\hypertarget{plotting-the-mean-sea-ice-concentration-used-on-january-21st-2022-by-6-snow-petrels-from-sv.}{%
\subsection{Plotting the mean sea ice concentration used on January
21st, 2022 by 6 snow petrels from
SV.}\label{plotting-the-mean-sea-ice-concentration-used-on-january-21st-2022-by-6-snow-petrels-from-sv.}}

\hypertarget{concentration-as-read-from-the-sea-ice-chart-found-here-aari}{%
\subsubsection{\texorpdfstring{Concentration as read from the sea ice
chart found here:
\href{http://ice.aari.aq/antice/2022/01/20220121_nic/}{AARI}}{Concentration as read from the sea ice chart found here: AARI}}\label{concentration-as-read-from-the-sea-ice-chart-found-here-aari}}

\begin{Shaded}
\begin{Highlighting}[]
\CommentTok{\# Sea ice chart concentration (observed}
\ControlFlowTok{for}\NormalTok{(i }\ControlFlowTok{in} \DecValTok{1}\SpecialCharTok{:}\DecValTok{6}\NormalTok{)\{}
  \CommentTok{\#for(i in 2:length(bird))\{}
  \CommentTok{\#this makes the data {-} using the range of the minumum and maximum sst for the entire tracked period}
\NormalTok{  tmpSV}\OtherTok{=}\FunctionTok{density}\NormalTok{(comp}\SpecialCharTok{$}\NormalTok{CT[comp}\SpecialCharTok{$}\NormalTok{BIRDID }\SpecialCharTok{==}\NormalTok{ birdSV[i]], }\AttributeTok{bw =} \DecValTok{8}\NormalTok{, }\AttributeTok{na.rm=}\NormalTok{T, }\AttributeTok{from=}\FunctionTok{min}\NormalTok{(comp}\SpecialCharTok{$}\NormalTok{CT,}\AttributeTok{na.rm=}\NormalTok{T), }\AttributeTok{to=}\FunctionTok{max}\NormalTok{(comp}\SpecialCharTok{$}\NormalTok{CT,}\AttributeTok{na.rm=}\NormalTok{T))}
  \CommentTok{\#x here is the range of SST (regularised to intervals)}
  \CommentTok{\#note {-} str = structure function}
  \FunctionTok{print}\NormalTok{(tmpSV}\SpecialCharTok{$}\NormalTok{bw)}
  \CommentTok{\#saves the x value}
\NormalTok{  CTmy.xSV}\OtherTok{=}\NormalTok{tmpSV}\SpecialCharTok{$}\NormalTok{x}
  \CommentTok{\#pulls out the y value and standardises it to sum to 1{-} y is the density for individuals in the loop}
\NormalTok{  tmpSV }\OtherTok{=}\NormalTok{ tmpSV}\SpecialCharTok{$}\NormalTok{y}\SpecialCharTok{/}\FunctionTok{sum}\NormalTok{(tmpSV}\SpecialCharTok{$}\NormalTok{y)}
  \CommentTok{\#for the first itteration {-}renaming standardised y values then on subsequent loops {-} this makes a matrix of a column for each individual}
  \ControlFlowTok{if}\NormalTok{(i}\SpecialCharTok{==}\DecValTok{1}\NormalTok{)\{tmpSV2 }\OtherTok{=}\NormalTok{ tmpSV\}}\ControlFlowTok{else}\NormalTok{\{tmpSV2}\OtherTok{=}\FunctionTok{cbind}\NormalTok{(tmpSV2,tmpSV)\}}
\NormalTok{\}}
\end{Highlighting}
\end{Shaded}

\begin{verbatim}
## [1] 8
## [1] 8
## [1] 8
## [1] 8
## [1] 8
## [1] 8
\end{verbatim}

\begin{Shaded}
\begin{Highlighting}[]
\CommentTok{\#takes the matrix of a column for each bird {-} across the rows {-} take the mean (1 = apply across rows)}
\NormalTok{CTtmpSV2}\OtherTok{=}\FunctionTok{apply}\NormalTok{(tmpSV2,}\DecValTok{1}\NormalTok{,mean)}
\FunctionTok{plot}\NormalTok{(CTmy.xSV,CTtmpSV2, }\StringTok{\textquotesingle{}l\textquotesingle{}}\NormalTok{)}
\FunctionTok{points}\NormalTok{(CTmy.xSV,tmpSV2[,}\DecValTok{1}\NormalTok{],}\StringTok{\textquotesingle{}l\textquotesingle{}}\NormalTok{,}\AttributeTok{col=}\StringTok{\textquotesingle{}grey\textquotesingle{}}\NormalTok{)}
\end{Highlighting}
\end{Shaded}

\includegraphics{07_Compare_SIC_to_Sea_Ice_Charts_files/figure-latex/unnamed-chunk-3-1.pdf}

\begin{Shaded}
\begin{Highlighting}[]
\CommentTok{\# Example plot {-} creates a blank plot with full range {-} n is base r for blank plot}
\FunctionTok{plot}\NormalTok{(}\FunctionTok{range}\NormalTok{(CTmy.xSV),}\FunctionTok{range}\NormalTok{(tmpSV2),}\StringTok{\textquotesingle{}n\textquotesingle{}}\NormalTok{, }\AttributeTok{xlab =} \StringTok{"AMSR2 vs Charts on Jan 21 (SV)"}\NormalTok{, }\AttributeTok{ylab =} \StringTok{"Density"}\NormalTok{)}
\ControlFlowTok{for}\NormalTok{(i }\ControlFlowTok{in} \DecValTok{1}\SpecialCharTok{:}\FunctionTok{ncol}\NormalTok{(tmpSV2)) }\FunctionTok{points}\NormalTok{(CTmy.xSV,tmpSV2[,i],}\StringTok{\textquotesingle{}l\textquotesingle{}}\NormalTok{,}\AttributeTok{col=}\StringTok{\textquotesingle{}grey\textquotesingle{}}\NormalTok{)}
\CommentTok{\#points(CTmy.xSV,CTtmpSV2, \textquotesingle{}l\textquotesingle{}, col = "red")}
\FunctionTok{points}\NormalTok{(CTmy.xSV,CTtmpSV2, }\AttributeTok{type =} \StringTok{\textquotesingle{}l\textquotesingle{}}\NormalTok{, }\AttributeTok{lwd =} \DecValTok{2}\NormalTok{,  }\AttributeTok{lty =} \DecValTok{1}\NormalTok{, }\AttributeTok{col =} \StringTok{"red"}\NormalTok{)}
\end{Highlighting}
\end{Shaded}

\includegraphics{07_Compare_SIC_to_Sea_Ice_Charts_files/figure-latex/unnamed-chunk-3-2.pdf}

\hypertarget{concentration-as-read-from-the-sea-ice-chart-found-here-amsr2}{%
\subsubsection{\texorpdfstring{Concentration as read from the sea ice
chart found here:
\href{https://seaice.uni-bremen.de/databrowser/\#day=6\&month=2\&year=2023\&img=\%7B\%22image\%22\%3A\%22image-1\%22\%2C\%22product\%22\%3A\%22AMSR\%22\%2C\%22type\%22\%3A\%22visual\%22\%2C\%22region\%22\%3A\%22Antarctic3125\%22\%7D}{AMSR2}}{Concentration as read from the sea ice chart found here: AMSR2}}\label{concentration-as-read-from-the-sea-ice-chart-found-here-amsr2}}

\begin{Shaded}
\begin{Highlighting}[]
\CommentTok{\# Passive Microwave Concentration (AMSR2)}
\ControlFlowTok{for}\NormalTok{(i }\ControlFlowTok{in} \DecValTok{1}\SpecialCharTok{:}\DecValTok{6}\NormalTok{)\{}
  \CommentTok{\#for(i in 2:length(bird))\{}
  \CommentTok{\#this makes the data {-} using the range of the minumum and maximum sst for the entire tracked period}
\NormalTok{  sicSV}\OtherTok{=}\FunctionTok{density}\NormalTok{(comp}\SpecialCharTok{$}\NormalTok{concentration[comp}\SpecialCharTok{$}\NormalTok{BIRDID }\SpecialCharTok{==}\NormalTok{ birdSV[i]], }\AttributeTok{bw =} \DecValTok{10}\NormalTok{, }\AttributeTok{na.rm=}\NormalTok{T, }\AttributeTok{from=}\FunctionTok{min}\NormalTok{(comp}\SpecialCharTok{$}\NormalTok{concentration,}\AttributeTok{na.rm=}\NormalTok{T), }\AttributeTok{to=}\FunctionTok{max}\NormalTok{(comp}\SpecialCharTok{$}\NormalTok{concentration,}\AttributeTok{na.rm=}\NormalTok{T))}
  \CommentTok{\#x here is the range of SST (regularised to intervals)}
  \CommentTok{\#note {-} str = structure function}
  \FunctionTok{print}\NormalTok{(sicSV}\SpecialCharTok{$}\NormalTok{bw)}
  \CommentTok{\#saves the x value}
\NormalTok{  concmy.xSV}\OtherTok{=}\NormalTok{sicSV}\SpecialCharTok{$}\NormalTok{x}
  \CommentTok{\#pulls out the y value and standardises it to sum to 1{-} y is the density for individuals in the loop}
\NormalTok{  sicSV }\OtherTok{=}\NormalTok{ sicSV}\SpecialCharTok{$}\NormalTok{y}\SpecialCharTok{/}\FunctionTok{sum}\NormalTok{(sicSV}\SpecialCharTok{$}\NormalTok{y)}
  \CommentTok{\#for the first itteration {-}renaming standardised y values then on subsequent loops {-} this makes a matrix of a column for each individual}
  \ControlFlowTok{if}\NormalTok{(i}\SpecialCharTok{==}\DecValTok{1}\NormalTok{)\{sicSV2 }\OtherTok{=}\NormalTok{ sicSV\}}\ControlFlowTok{else}\NormalTok{\{sicSV2}\OtherTok{=}\FunctionTok{cbind}\NormalTok{(sicSV2,sicSV)\}}
\NormalTok{\}}
\end{Highlighting}
\end{Shaded}

\begin{verbatim}
## [1] 10
## [1] 10
## [1] 10
## [1] 10
## [1] 10
## [1] 10
\end{verbatim}

\begin{Shaded}
\begin{Highlighting}[]
\CommentTok{\#takes the matrix of a column for each bird {-} across the rows {-} take the mean (1 = apply across rows)}
\NormalTok{conctmpSV2}\OtherTok{=}\FunctionTok{apply}\NormalTok{(sicSV2,}\DecValTok{1}\NormalTok{,mean)}
\FunctionTok{plot}\NormalTok{(concmy.xSV,conctmpSV2, }\StringTok{\textquotesingle{}l\textquotesingle{}}\NormalTok{)}
\FunctionTok{points}\NormalTok{(concmy.xSV,sicSV2[,}\DecValTok{1}\NormalTok{],}\StringTok{\textquotesingle{}l\textquotesingle{}}\NormalTok{,}\AttributeTok{col=}\StringTok{\textquotesingle{}grey\textquotesingle{}}\NormalTok{)}
\end{Highlighting}
\end{Shaded}

\includegraphics{07_Compare_SIC_to_Sea_Ice_Charts_files/figure-latex/unnamed-chunk-5-1.pdf}

\begin{Shaded}
\begin{Highlighting}[]
\CommentTok{\# Example plot {-} creates a blank plot with full range {-} n is base r for blank plot}
\FunctionTok{plot}\NormalTok{(}\FunctionTok{range}\NormalTok{(concmy.xSV),}\FunctionTok{range}\NormalTok{(sicSV2),}\StringTok{\textquotesingle{}n\textquotesingle{}}\NormalTok{, }\AttributeTok{xlab =} \StringTok{"AMSR2 vs Charts on Jan 21 (SV)"}\NormalTok{, }\AttributeTok{ylab =} \StringTok{"Density"}\NormalTok{)}
\ControlFlowTok{for}\NormalTok{(i }\ControlFlowTok{in} \DecValTok{1}\SpecialCharTok{:}\FunctionTok{ncol}\NormalTok{(sicSV2)) }\FunctionTok{points}\NormalTok{(concmy.xSV,sicSV2[,i],}\StringTok{\textquotesingle{}l\textquotesingle{}}\NormalTok{,}\AttributeTok{col=}\StringTok{\textquotesingle{}grey\textquotesingle{}}\NormalTok{)}
\FunctionTok{points}\NormalTok{(concmy.xSV,conctmpSV2, }\StringTok{\textquotesingle{}l\textquotesingle{}}\NormalTok{, }\AttributeTok{col =} \StringTok{"darkcyan"}\NormalTok{)}
\FunctionTok{points}\NormalTok{(concmy.xSV,conctmpSV2, }\AttributeTok{type =} \StringTok{\textquotesingle{}l\textquotesingle{}}\NormalTok{, }\AttributeTok{lwd =} \DecValTok{2}\NormalTok{,  }\AttributeTok{lty =} \DecValTok{1}\NormalTok{, }\AttributeTok{col =} \StringTok{"darkcyan"}\NormalTok{)}
\end{Highlighting}
\end{Shaded}

\includegraphics{07_Compare_SIC_to_Sea_Ice_Charts_files/figure-latex/unnamed-chunk-5-2.pdf}

\begin{Shaded}
\begin{Highlighting}[]
\CommentTok{\# Plot both}

\DocumentationTok{\#\#\# combine}

\CommentTok{\#CT range is bigger so use that to set extent of plot}
\FunctionTok{plot}\NormalTok{(}\FunctionTok{range}\NormalTok{(CTmy.xSV),}\FunctionTok{range}\NormalTok{(tmpSV2),}\StringTok{\textquotesingle{}n\textquotesingle{}}\NormalTok{, }\AttributeTok{xlab =} \StringTok{"AMSR2 vs Sea Ice Chart concentration for SV birds on 21/01/22"}\NormalTok{, }\AttributeTok{ylab =} \StringTok{"Density"}\NormalTok{)}
\ControlFlowTok{for}\NormalTok{(i }\ControlFlowTok{in} \DecValTok{1}\SpecialCharTok{:}\FunctionTok{ncol}\NormalTok{(sicSV2)) }\FunctionTok{points}\NormalTok{(concmy.xSV,sicSV2[,i],}\StringTok{\textquotesingle{}l\textquotesingle{}}\NormalTok{,}\AttributeTok{col=}\StringTok{\textquotesingle{}cadetblue\textquotesingle{}}\NormalTok{)}
\ControlFlowTok{for}\NormalTok{(i }\ControlFlowTok{in} \DecValTok{1}\SpecialCharTok{:}\FunctionTok{ncol}\NormalTok{(tmpSV2)) }\FunctionTok{points}\NormalTok{(CTmy.xSV,tmpSV2[,i],}\StringTok{\textquotesingle{}l\textquotesingle{}}\NormalTok{,}\AttributeTok{col=}\StringTok{\textquotesingle{}coral\textquotesingle{}}\NormalTok{)}
\FunctionTok{points}\NormalTok{(CTmy.xSV,CTtmpSV2, }\StringTok{\textquotesingle{}l\textquotesingle{}}\NormalTok{, }\AttributeTok{col =} \StringTok{"red"}\NormalTok{, }\AttributeTok{lwd =} \DecValTok{2}\NormalTok{)}
\FunctionTok{points}\NormalTok{(concmy.xSV,conctmpSV2, }\StringTok{\textquotesingle{}l\textquotesingle{}}\NormalTok{, }\AttributeTok{col =} \StringTok{"darkcyan"}\NormalTok{, }\AttributeTok{lwd =} \DecValTok{2}\NormalTok{)}
\FunctionTok{legend}\NormalTok{(}\DecValTok{25}\NormalTok{, }\FloatTok{0.0085}\NormalTok{, }\AttributeTok{legend=}\FunctionTok{c}\NormalTok{(}\StringTok{"Passive Microwave (AMSR2)"}\NormalTok{, }\StringTok{"Sea Ice Chart"}\NormalTok{),}
       \AttributeTok{col=}\FunctionTok{c}\NormalTok{(}\StringTok{"cadetblue"}\NormalTok{, }\StringTok{"red"}\NormalTok{), }\AttributeTok{lty=}\DecValTok{1}\NormalTok{, }\AttributeTok{cex=}\FloatTok{0.8}\NormalTok{, }\AttributeTok{lwd =} \DecValTok{2}\NormalTok{)}
\end{Highlighting}
\end{Shaded}

\includegraphics{07_Compare_SIC_to_Sea_Ice_Charts_files/figure-latex/unnamed-chunk-6-1.pdf}

\end{document}
